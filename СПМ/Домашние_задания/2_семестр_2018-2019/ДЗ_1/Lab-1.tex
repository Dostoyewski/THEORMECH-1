\documentclass{article}
  \usepackage[utf8]{inputenc}
  \usepackage[russian]{babel}
  \usepackage[left=2cm,right=2cm,
  top=2cm,bottom=2cm,bindingoffset=0cm]{geometry}
\usepackage{graphicx}
\usepackage{amsmath}
\usepackage{float}
\usepackage{listings}
\usepackage{mcode}
\usepackage{url,textcomp}
\author{Кондратенко Федор, гр 13632/1}
\title{Отчет по домашнему заданию №1 второго семестра 2018-2019 учебного года}
\date{\today}
\setlength{\parindent}{0pt}
\setlength{\parskip}{5pt plus 2pt minus 1pt}
\frenchspacing
\begin{document}
\maketitle
\section*{Задание 1:  Построение графика функции с использованием блоков Simulink}
Составить блок схему функции и получить ее график двумя способами:
\begin{enumerate} 
    \item {с помощью блока Scope;}
    \item {путем передачи данных в рабочую область и применения приложения PLOTS.}
\end{enumerate}
Функция: $$ y = \frac{lg(x^2 - 1)}{log_5(ax^2 - b)}, a = 1.1, b = 0.09 $$

\begin{figure}[!h]
	
	\centering
	
	\includegraphics[width=0.7\linewidth]{SC_1.png}
	
	\caption{Блок-схема 1}
	
	\label{fig:mpr}
	
\end{figure}
~\\
~\\\\
~\\
~\\\\
~\\
~\\\\
~\\
~\\\\

Результат моделирования на участке 0..40 секунд:\\

\begin{figure}[!h]
	
	\centering
	
	\includegraphics[width=0.7\linewidth]{Graph_1.jpg}
	
	\caption{График 1}
	
	\label{fig:mpr}
	
\end{figure}


Построение тех же данных через plot:

\begin{figure}[h!]
	\centering
	\includegraphics[height=0.4\textheight]{Graph_2.jpg}
	\caption{График 2}
	\label{fig:mpr}
\end{figure}
\section*{Задание 2: Решение нелинейного алгебраического уравнения}
Составить блок схему и получить численное решение нелинейного уравнения. Результат вывести на цифровой дисплей.\\
Уравнение: $$x^4+2x^3-x-1=0$$

\begin{figure}[!h]
	\centering
	\includegraphics[width=0.9\linewidth]{SC_2}
	\caption{Блок-схема 2}
	\label{fig:mpr}
\end{figure}

\section*{Задание 3: Решение систем алгебраических уравнений}
Cоставить блок схему решения системы 4-х линейных уравнений и получить ее численное решение. Решить данную систему символьно и сопоставить результаты.~\\

$$\begin{cases}
	2x_1+x_2-5x_3+x_4=-4\\
	x_1-3x_2-6x_4=-7\\
	2x_2-x_3+2x_4=2\\
	x_1+4x_2-7x_3+6x_4=-2\\
\end{cases}$$
\begin{figure}[!h]
	\centering
	\includegraphics[width=1\linewidth]{SC_3}
	\caption{Блок-схема 3}
	\label{fig:sc3}
\end{figure}
Численное решение в Simulink совпало с решением в Matlab.
~\\
~\\\\
~\\
~\\\\ Для проверки решения использовался следующий код:
\begin{lstlisting}
A = [2 1 -5 1;
    1 -3 0 -6;
    0 2 -1 2;
    1 4 -7 6];
B = [-4; -7; 2; -2];
X = linsolve(A, B)
fprintf("Solution: x1 = %f, x2 = %f, x3 = %f, x4 = %f\n", X(1), X(2), X(3), X(4));
\end{lstlisting}

Вывод в консоль:
\begin{lstlisting}
X =

    2.0000
    1.0000
    2.0000
    1.0000

Solution: x1 = 2.000000, x2 = 1.000000, x3 = 2.000000, x4 = 1.000000

\end{lstlisting}
~\\
~\\\\ 
Решение системы:
$$\begin{cases}
	x_1=2\\
	x_2=1\\
	x_3=2\\
	x_4=1\\
\end{cases}$$
\end{document}
 